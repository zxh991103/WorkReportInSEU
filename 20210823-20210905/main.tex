\documentclass[a4paper]{article}

\usepackage[utf8]{inputenc}
% \usepackage[fontset=ubuntu]{ctex}
\usepackage{ctex}
\usepackage[ruled,linesnumbered]{algorithm2e}
\usepackage{xcolor}
\usepackage{amsmath}
\usepackage{graphicx}
\usepackage{listings}
\usepackage{verbatim}
\usepackage{bigints}
\usepackage{amsmath}
\usepackage{float}
\usepackage{titling}
\usepackage[colorinlistoftodos]{todonotes}
\usepackage[colorlinks,linkcolor=red]{hyperref}

\definecolor{codegreen}{rgb}{0,0.6,0}
\definecolor{codegray}{rgb}{0.5,0.5,0.5}
\definecolor{codepurple}{rgb}{0.58,0,0.82}
\definecolor{backcolour}{rgb}{0.95,0.95,0.92}

\lstdefinestyle{mystyle}{
    backgroundcolor=\color{backcolour},   
    commentstyle=\color{codegreen},
    keywordstyle=\color{magenta},
    numberstyle=\tiny\color{codegray},
    stringstyle=\color{codepurple},
    basicstyle=\ttfamily\footnotesize,
    breakatwhitespace=false,         
    breaklines=true,                 
    captionpos=b,                    
    keepspaces=true,                 
    numbers=left,                    
    numbersep=5pt,                  
    showspaces=false,                
    showstringspaces=false,
    showtabs=false,                  
    tabsize=2
}

\lstset{style=mystyle}

\title{8.23-9.5 周报}

\author{赵晓辉}

\date{\today}

\begin{document}
\maketitle

\section{写在前面}

这个双周主要完成了cs231n课程的Lecture 1至Lecture 6,主要内容包括距离函数、KNN、SVM、损失函数及优化、BP算法、CNN架构、非线性激活函数以及神经网络的参数优化等,并完成cs231n assignment1。课程概要笔记及assignment将在\url{https://github.com/zxh991103/cs231NOTE}持续跟踪。




\section{Lec 1-6课程概要}
\subsection{距离函数}

$L_1$ Distance
$$d_1(I_1,I_2)=\sum_p|I^p_1-I^p_2|$$

$L_2$ Distance
$$d_2(I_1,I_2) = \sqrt{\sum_p(I^p_1-I^p_2)^2}$$

\subsection{KNN}

计算测试样本与所有训练集样本之间的距离值,并根据K值投票选举出最相似的标签。

\subsection{SVM}

计算能够划分训练集样本且距离最大的超平面。
$$w\cdot x + b=0$$

\subsection{损失函数}
损失函数评估模型预测值与模型真实值之间的差异性,我们要将其最小化。
对于给定的训练集${(x_i,y_i)}^N_{i=1}$,我们有损失函数:
$$L =\frac{1}{N} \sum_i L_i(f(x_i,W),y_i)$$

对于Multi-SVM,我们有损失函数,即hinge loss:
$$s=f(x_i,W)$$
$$L_i=\sum_{j\neq y_i}max(0,s_j-s_{y_i}+1)$$

对于softmax loss:
$$L_i = -log (\frac{e^{s_{y_i}}}{\sum_j e^{s_{j}}})$$

\subsection{正则化}

根据奥卡姆剃刀原则,模型越简单越符合实际,所以我们将正则惩罚项
加在损失函数上。

$$L =\frac{1}{N} \sum_i L_i(f(x_i,W),y_i)+λ R(W)$$

L1
$$R(W)=\sum_k \sum_l |W_{k,l}|$$

L2
$$R(W)=\sum_k \sum_l W_{k,l}^2$$

Elastic
$$R(W)=\sum_k \sum_l \beta W_{k,l}^2+|W_{k,l}|$$
\subsection{BP算法}
链式法则:
$$\frac{\partial f}{\partial y}=\frac{\partial f}{\partial q}\frac{\partial q}{\partial y}$$

故在计算损失函数对于参数的梯度值时,我们应当将本地梯度值与上游回传梯度值相乘。

此时,我们也可以发现Relu函数,即max gate中只有前向传播计算中的正值能影响下游。

A vectorized example:
$$f(x,W) =||W ⋅ x||^2 = \sum_{i=1}^n(W ⋅ x)_i^2$$
$$q = W \cdot x$$
$$\nabla_W f = 2q \cdot x^T$$


\subsection{NN}
假设我们将神经网络的计算图表示为:
$$f = Softmax(W_2 Relu(W_1 x))$$

神经元A拥有$W_1$,能够具有识别出来100种特征的功能,比如识别出马的左脸或者右脸、车头或者车位。
而神经元B拥有$W_2$,其功能就在于将马的左脸或右脸合并为马的特征,将车头或车尾合并成车的特征,从而进行识别。

\subsection{CNN}

卷积层:

假设有32*32*3的图片,卷积核w 5*5*3 ,以及偏置b,
卷积后我们获得28*28*1的矩阵,其中1时卷积核的数量。
卷积公式为(每位相乘再求和):
$$
f[x,y] * g[x,y] = \sum_{n_1=-\infty}^{\infty } \sum_{n_2=-\infty}^{\infty } f[x,y] * g[x-n_1,y-n_2]
$$

步长(stride):

假设我们有7*7的输入,3*3的卷积核,2的步长,最后的输出为3*3。
此时outputsize = $\frac{(N-F)}{stride}+1$

填充(Pad):

图像四周补充0,来防止在深层卷积时张量过小。此时,outputsize = $\frac{(N-F+2P)}{stride}+1$。

Example:

input volume 32*32*3,10 5*5 filters (include 3 depth), stride 1 ,pad 2 ,wo have 760 parameters ( 10 *( 5 * 5 *3 +1 bias)=760)

pooling layer:
相当于下采样。

maxpooling:

一般,每一个池化filter具有和步长相同的大小以避免overlap.

例如,
$$\begin{matrix}  
    1 & 1 & 2 & 4 \\
    5 & 6 & 7 & 8 \\
    3 & 2 & 1 & 0 \\
    1 & 2 & 3 & 4
\end{matrix}$$

我们使用2*2的filter和2的stride,maxpooling后变为:
$$\begin{matrix}  
    6 & 8 \\
    3 & 4  \\
   
\end{matrix}$$



\subsection{激活函数}
若全部线性连接则等同于一个线性连接,所以网络中需要非线性的激活函数变换。

sigmod 

$$
\sigma(x)=\frac{1}{1+e^{-x}}
$$

tanh

$$tanh(x)$$

Relu

$$max(0,x)$$

LeakyRelu
$$max(0.1x,x)$$

Maxout
$$
\max \left(w_{1}^{T} x+b_{1}, w_{2}^{T} x+b_{2}\right)
$$

Elu
$$
\begin{cases}
x & x \geq 0 \\ 
\alpha\left(e^{x}-1\right) & x<0
\end{cases}
$$

SELU
$$
f(x)= \begin{cases}\lambda x & \text { if } x>0 \\ \lambda \alpha\left(e^{x}-1\right) & \text { otherwise }\end{cases}
$$


softmax的问题:
\begin{itemize}
    \item x过大或过小,本地梯度接近0,使得与上游梯度乘积也接近于0,更新缓慢。
    \item x的值恒正或恒负,本地梯度总是大于0的,造成w的移动时锯齿状的,接近最优点放缓。
    
\end{itemize}

relu的问题:
\begin{itemize}
    \item 若$w\cdot x+b$总是负的,则本地梯度为0,造成参数不更新。
    
\end{itemize}


\subsection{数据处理}
\begin{lstlisting}[language=Python, caption=normalization]
# ZERO-CENTER
X -= np.mean(X,axis = 0)
# normalize
X /= np.std(X,axis = 0)
\end{lstlisting}

\subsection{参数初始化}
Naive:为参数初始化小随机数。但是随着网络深度的增加,本地梯度与上游梯度相乘之后接近零,学习十分缓慢。

Xavier:
\begin{lstlisting}[language=Python, caption=Xavier]
W = np.random.randn(dim_in,dim_out)/np.sqrt(dim_in)
\end{lstlisting}

原因:

we want Var(y) = Var($x_i$) , and we have 
$$y = \sum_{i=1}^{Din} x_i w_i$$

and we assume that  every x has same var. so we have 

$$
var(y) = Din \times var(x) \times var(w_i)
$$

and obviously initial $w_i ~ N(0,1)$ , we make $\frac{w_i}{\sqrt{Din}}$ to achieve the var is $\frac{1}{Din}$

Kaiming/MSRA:
\begin{lstlisting}[language=Python, caption=MSRA]
    W = np.random.randn(dim_in,dim_out)*np.sqrt(2/dim_in)
\end{lstlisting}

\subsection{Batch Normalization}

we have the input  x like N$\times$D 
$$
\widehat{x}^{(k)}=\frac{x^{(k)}-\mathrm{E}\left[x^{(k)}\right]}{\sqrt{\operatorname{Var}\left[x^{(k)}\right]}}
$$

so that we have:
$$
\mu_{j}=\frac{1}{N} \sum_{i=1}^{N} x_{i, j}
$$
$$
\sigma_{j}^{2}=\frac{1}{N} \sum_{i=1}^{N}\left(x_{i, j}-\mu_{j}\right)^{2}
$$
$$
\hat{x}_{i, j}=\frac{x_{i, j}-\mu_{j}}{\sqrt{\sigma_{j}^{2}+\varepsilon}}
$$


the net is supposed to learn $\gamma \in R^D$  and  $\beta\in R^D$ 
$$
y_{i, j}=\gamma_{j} \hat{x}_{i, j}+\beta_{j}
$$

BN 层经常用于全连接或卷积层后。

\subsection{Norm For Conv.}

batch norm , 对于每一个batch中的每个channel取平均得到 (1× 1 × C)

layer norm ,对于所有的batch我们取一个平均的图片 (H × W× C)

instance norm ,对于所有的batch我们将取一个平均的单通道图片 (H × W ×1)

group norm , 对于所有batch,我们将channel分为k个组,在每个组山上取平均值得到  (H × W × k)

\section{Assignment1}



\end{document}